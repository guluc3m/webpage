\documentclass[11pt,oneside,onecolumn,a4paper]{article} \usepackage[utf8]{inputenc} \usepackage[spanish]{babel} \usepackage{enumerate} \pagestyle{plain}

\newcounter{articulos} \setcounter{articulos}{1} \newcounter{capitulos} \setcounter{capitulos}{1} \newcounter{titulos} \setcounter{titulos}{1}

\newcommand{\articulo}{\medskip \noindent \emph{Artículo \arabic{articulos} } \smallskip \addtocounter{articulos}{1}} \newcommand{\titulo}[1]{\medskip \noindent \emph{Título \roman{titulos}:} #1 \addcontentsline{toc}{section}{Título \roman{titulos}: #1} \smallskip \addtocounter{titulos}{1}} \newcommand{\capitulo}[1]{\medskip \noindent \emph{Capítulo \roman{capitulos}:} #1 \addcontentsline{toc}{subsection}{Capítulo \roman{capitulos}: #1} \smallskip \addtocounter{capitulos}{1}}

\title{Estatutos de la Asociación\\ Grupo de Usuarios de Linux\\de la\\Universidad Carlos III de Madrid \\ (G.U.L.)} \author{} \date{12 de Diciembre de 2008}

\begin{document} \sloppy \maketitle

%\begin{abstract} %  Este documento recoge los Estatutos de la Asociación denominada Grupo de Usuarios de Linux de la Universidad Carlos III de Madrid, presentados ante el Registro de Asociaciones de la Comunidad de Madrid. Su difusión es libre siempre que no se realicen modificaciones sobre el original. %\end{abstract}

\newpage

\tableofcontents \newpage

\pagebreak[1]

\section*{\titulo{Denominación y Domicilio}}

%\subsection*{De la denominación de la Asociación} %\addcontentsline{toc}{subsection}{De la denominación de la Asociación}

\articulo

Se constituye en el campus de Leganés de la Universidad Carlos III de Madrid la asociación denominada Grupo de Usuarios de Linux de la Universidad Carlos III de Madrid, al amparo de lo previsto en el artículo 22 de la Constitución Española de 1978, lo establecido en la Ley Orgánica 1/2002 de 22 de Marzo, el Real Decreto 1497/2003, la Ley Orgánica 1/82 del 5 de Mayo, el Reglamento de asociaciones de la Universidad Carlos III de Madrid y demás disposiciones legales.

\articulo

El régimen de la Asociación se determinará por lo dispuesto en los presentes estatutos, si bien éstos podrán ser completados por un Reglamento de Funcionamiento Interno.

\articulo

El domicilio social de la Asociación radicará en el campus Leganés de Universidad Carlos III de Madrid, ubicado en\\ Avenida de la Universidad, número 30\\ 28911, Leganés\\ Madrid.\\ La junta directiva podrá acordar el cambio de domicilio, de lo que dará comunicación a las autoridades pertinentes.

\articulo

La duración de la Asociación será por tiempo indefinido, carece de ánimo de lucro y el ámbito territorial principal de sus actividades será la Comunidad Autónoma de Madrid.

\articulo

En todo cuanto no esté previsto en los presentes Estatutos se aplicará la vigente Ley de Asociaciones del 22 de marzo de 2002 y demás disposiciones complementarias.

\section*{\titulo{Objetivos, actividades y medios}}

\articulo

Linux es el núcleo de un sistema operativo informático creado inicialmente por Linus Torvalds y posteriormente ampliado y modificado por miles de personas de todo el mundo. Este proyecto informático tiene una declaración de derechos de copia que cumplen con los preceptos establecidos como Software Libre, y se adopta su nombre como ejemplo representativo de este.

El Software Libre se define como todo programa informático con una declaración de derechos de copia que permita: \begin{itemize} \item{Usarse para cualquier propósito.} \item{Estudiar el programa y adaptarlo, para lo que es necesario acceder al código fuente del programa.} \item{Distribuir copias, modificadas o sin modificar.} \item{Mejorar el programa y hacer públicas las mejoras.} \end{itemize}

Se entiende como código fuente el programa tal y como lo escribieron los autores, antes de ser tratado de manera alguna.

Con el fin de divulgar y facilitar el acceso al Software Libre a los usuarios informáticos y de coordinar, apoyar y dar organización surge la iniciativa de crear una asociación, sin ánimo de lucro y de ámbito autonómico, al amparo de las disposiciones legales vigentes mencionadas en el Artículo 1.

\articulo

Los fines de la Asociación serán:

\begin{itemize}

  \item Divulgación y promoción de las tecnologías de información en general y del Software Libre en particular. \item Divulgación y promoción de los sistemas operativos basados en Software Libre como ejemplo de plataforma informática libre. \item Organización y apoyo a los grupos de usuarios y desarrolladores de Software libre. \item Servir de medio de promoción, intercambio de experiencias y divulgación del Software Libre, entre alumnos y profesores de los distintos niveles de la enseñanza, especialmente universitaria. \item Promocionar el acceso al Software Libre y las tecnologías de la información. \item Promover actividades que fomenten el acceso a las tecnologías de la información y al Software Libre.

  \end{itemize}

  \section*{\titulo{Estructura organizativa}}

  \subsection*{\capitulo{De la Asamblea General}}

  \articulo

  La Asamblea General es el órgano supremo de la asociación y está compuesta por todos los socios de la Asociación.

  \articulo

  La asamblea general se reunirá, al menos, una vez al año y, con carácter extraordinario, tantas veces como solicite el presidente, la Junta Directiva, o un mínimo de un tercio de los miembros de la asociación.

  La asistencia a la Asamblea General será personal o por representación legal o voluntaria bastando para acreditar ésta un escrito o documento firmado por el socio y entregado a un asistente de su confianza.

  Siempre que, a juicio unánime de los miembros de la junta directiva, la tecnología garantice por medios telemáticos los mismos derechos que la presencia física a una asamblea, será aceptada la asistencia por medios telemáticos.

  \articulo 

  No existirá un quorum mínimo para constituir la asamblea, si bien será necesario que asista, al menos, el Presidente y el Secretario o, en su defecto, cualesquiera dos miembros de la Junta Directiva. Estos ejercerán de presidente y secretario de la asamblea.

  \articulo

  Entre la convocatoria y el día señalado para la celebración de una sesión de la Asamblea General, ordinaria o extraordinaria, habrán de mediar, en primera convocatoria, al menos siete días naturales.  En esta comunicación deberá constar así mismo la fecha en que, si procediera, tendrá lugar la sesión de la Asamblea en segunda convocatoria, sin que entre una y otra pueda mediar un plazo inferior a media hora ni superior a ocho días hábiles.

  \articulo

  Son competencias de la asamblea general:

  \begin{itemize}

    \item Elegir a los miembros de la junta directiva. \item Aprobar el estado de cuentas del ejercicio anterior. \item Aprobar las modificaciones de los estatutos propuestas por la junta directiva. \item Acordar, a propuesta de la junta directiva y mediante referéndum, la disolución de la asociación. \item Establecer, si así se decidiera, la cuota de socio.

    \end{itemize}

    \subsection*{\capitulo{De la junta directiva}}

    \articulo

    Sin perjuicio de las facultades de la asamblea general, la asociación estará representada por una Junta Directiva, que estará formada por:

    \begin{itemize}

      \item Presidente \item Secretario \item Coordinador de actividades \item Tesorero \item Un Vocal

      \end{itemize}

      La Junta Directiva podrá aumentar o disminuir el número de vocales para adecuarlo a las circunstancias y necesidades de la asociación, siendo uno el número mínimo.

      Todos los miembros de la Junta Directiva han de ser socios de la asociación.

      \articulo 

      Todo miembro de la asociación, mayor de edad, podrá presentarse como candidato a la Junta Directiva.

      %Este articulo debe ir en la asamblea, en al junta o en capitulo aparte? 

      \articulo 

      Los candidatos se presentarán entre la convocatoria de elecciones y media hora antes de la votación mediante comunicación a la Junta Directiva saliente. La Junta Directiva saliente podrá decidir otro medio de comunicación que resulte conveniente, habiendo de comunicarlo en la convocatoria de elecciones. Los candidatos a la junta directiva serán elegidos democráticamente como se detalla a continuación y mediante un método que garantice la libertad en las votaciones y los derechos de los socios.

      La Junta Directiva tendrá una vigencia máxima de un año, no existiendo límite de reelecciones.

      La Asamblea General, reunida en sesión extraordinaria, realizará simultáneamente una votación por cada cargo de la junta. Estas se realizarán bajo las siguientes normas:

      \begin{itemize}

        \item Cada candidato puede presentarse a más de un cargo. \item Las votaciones se ganarán por mayoría simple. \item En caso de que un candidato gane en más de un cargo, elegirá uno del que tomará posesión. Se procederá a una segunda vuelta en cada uno de los demás cargos en que hubiese ganado, siendo los candidatos los mismos que se habían presentado anteriormente menos este. \item El cargo de vicepresidente corresponderá al candidato, que no sea el presidente, que más votos haya obtenido. \item Los candidatos tomarán posesión de sus cargos en el periodo ineludible de una semana desde la votación. \item Si se produjese algún caso no contemplado en estas normas, los miembros ya electos de la Junta Directiva actuarán como árbitros para proponer una solución.

	\end{itemize}

	\articulo

	Es competencia de la junta directiva:

	\begin{itemize}

	  \item Proponer la modificación de los presentes estatutos única y exclusivamente por acuerdo adoptado en junta extraordinaria convocada al efecto, y con una mayoría igual o superior a tres cuartos de los miembros de la Junta Directiva. \item Proponer la disolución de la asociación, para posteriormente convocar a la Asamblea General, de forma extraordinaria, para que apruebe o rechace su decisión. Esta propuesta debe aprobarse por unanimidad en la Junta Directiva. \item Disposición y administración de bienes. \item Federación con otras Asociaciones, a tenor de la legislación vigente. \item Examen y aprobación de las cuentas por mayoría simple, previamente a su presentación a la Asamblea General, así como proponer la cuota de los socios. \item Creación y disolución de tantas secciones organizativas como considere, para un mejor funcionamiento de la asociación, así como el nombramiento y cese del coordinador de cada sección técnica. \item Elaboración y aprobación por mayoría del Reglamento de Funcionamiento Interno. \item Lectura y aprobación de las actas de la junta directiva anterior, de la memoria anual, del presupuesto de gastos e ingresos para el siguiente ejercicio y del plan de actividades, sin perjuicio de su aprobación posterior por la Asamblea General. \item Posibilidad de presentar oposición a las actividades organizadas o coordinadas por el Coordinador de Actividades. \item Potestad arbitral en los conflictos presentados en el normal funcionamiento de las Secciones Organizativas. \item Potestad arbitral en todo lo que no esté recogido en los presentes estatutos, el reglamento de funcionamiento interno, si lo hubiera, y demás normas y leyes a las que se acoge la asociación.

	  \end{itemize}

	  \articulo

	  La Junta Directiva se reunirá al menos una vez cada seis meses, pudiendo convocarse la reunión de cualquiera de estas maneras:

	  \begin{itemize}

	    \item A propuesta del Presidente \item A petición de un tercio de los miembros de la junta directiva, de lo que sus convocantes darán cuenta al secretario para que proceda ineludiblemente a su convocatoria.

	    \end{itemize}

	    El quorum necesario para la toma de decisiones será de la mitad de los miembros de la Junta Directiva. De todas las decisiones adoptadas se harán públicas posteriormente actas oficiales.

	    La aprobación de una votación de la Junta Directiva se hará por mayoría simple, excepto donde expresamente se diga lo contrario.

	    \articulo

	    La asistencia a la junta directiva será personal o mediante cualquier medio telemático que acuerden válido todos los miembros la Junta Directiva.

	    \articulo

	    En caso de vacante, por cualquier motivo, de cualquier miembro de la Junta Directiva, excepto del Presidente, el cargo vacante lo cubrirá la persona que ocupase el siguiente lugar en número de votos en los resultados de la última votación celebrada. Esta misma norma se aplicará en caso de renuncia del sustituto.  Si la lista de candidatos ya se hubiese agotado, la junta directiva podrá nombrar por unanimidad a cualquier otro socio que lo acepte voluntariamente. La Asamblea General deberá ratificarle en su puesto en Sesión Extraordinaria en un plazo máximo de un mes.

	    En caso de vacante del Presidente, será el Vicepresidente quien ocupará su cargo hasta las elecciones siguientes. En caso de que este también esté vacante ocupará el cargo el miembro de la junta no vacante con mayor número de votos en las últimas elecciones. La duración del mandato será por el tiempo que resta hasta las próximas elecciones a junta directiva, que se celebrarán de forma extraordinaria en un período máximo de un mes.

	    \articulo

	    Los miembros de la junta directiva podrán ser destituidos por:

	    \begin{itemize}

	      \item Desinterés manifiesto hacia la asociación, su cargo o las funciones que conlleva. \item Utilización de su cargo para el beneficio personal. \item Actuación claramente contraria a los objetivos de la asociación.

	      \end{itemize}

	      Los miembros de la Junta podrán ser destituidos de la siguiente forma:

	      La votación para la destitución se realizará a petición de, al menos, dos miembros de la Junta Directiva o de más de la mitad de los socios. Será necesario que tres cuartas partes del total de socios asistentes a la Asamblea General Extraordinaria ratifiquen la destitución para que esta tenga efecto. Existirá un plazo mínimo de 5 días desde la presentación de la moción de censura a la votación. Al haber quedado vacante el cargo, se aplicará el articulado sobre sustitución en caso de vacante. En caso de que el cargo destituido fuese también vicepresidente, este cargo pasará al siguiente miembro, que no sea el presidente, con mayor número de votos en las últimas elecciones.

	      \articulo

	      La Junta Directiva podrá acordar la creación o disolución de Secciones Organizativas que respondan a las necesidades específicas enmarcadas dentro de la Asociación. Estas estarán reguladas mediante el reglamento de funcionamiento interno, si lo hubiera.

	      Los miembros de cada sección organizativa podrán escoger democráticamente a un representante. Las secciones organizativas estarán representadas en la Junta Directiva mediante el Coordinador de Actividades.

	      \section*{\titulo{Órganos unipersonales}}

	      \articulo

	      Al presidente corresponde: \begin{itemize}

	        \item La representación legal y oficial de la asociación, ante toda clase de autoridades, organismos y particulares, tribunales de justicia, personas físicas y jurídicas, pudiendo delegar dicha representación. \item Convocar y presidir las reuniones de la Junta Directiva y la Asamblea General. \item Cumplir y hacer cumplir los acuerdos de la Junta Directiva y la Asamblea General. \item Velar por el exacto cumplimiento de los estatutos. \item Autorizar con su firma las actas, certificaciones y demás documentos oficiales de la asociación, y que normalmente expedirá el Secretario. \item Tener firma en la cuenta bancaria de la asociación. \item En el caso de empate en votaciones de los plenos de la Junta Directiva, ejercerá el derecho de Voto de Calidad. \item Evaluar periódicamente el trabajo realizado por sus colaboradores y por la asociación. \item Cualquier otra estipulada por el reglamento de funcionamiento interno.

		\end{itemize}

		\articulo

		El miembro de la junta que además ostente el cargo de Vicepresidente tendrá como función la sustitución temporal del Presidente en los casos de ausencia obligada de éste, enfermedad o dimisión.

		Además, podrá actuar por delegación expresa del Presidente en aquellas funciones que éste le asigne y sean de su competencia.

		Estas funciones se añadirán a las del cargo que ostente.

		\articulo

		Como Secretario tendrá a su cargo el funcionamiento administrativo de la asociación. Su misión será:

		\begin{itemize}

		  \item Tener bajo su responsabilidad el Archivo y custodia de todos los documentos y sellos de la asociación. \item Redactará las actas de las reuniones y asambleas, asistiendo al Presidente durante las mismas. \item Redactará la Memoria Anual y las actas de la asociación. \item Llevará al día un libro de socios, recibiendo y tramitando las solicitudes de altas y bajas de socios. \item Citar a los socios para las reuniones que se convoquen, según corresponda al tipo de reunión. \item Cualquier otra estipulada por el reglamento de funcionamiento interno.

		  \end{itemize}

		  \articulo

		  Al Coordinador de actividades le corresponde: \begin{itemize}

		    \item Dirigir, orientar, fomentar y coordinar todas las actividades realizadas por la Asociación. \item Informar a la Junta Directiva de los proyectos a realizar. Redactará una Memoria Explicativa Anual para la Junta Directiva y responderá a todas las cuestiones que ésta pueda plantear mientras ostente el cargo. \item Evaluar periódicamente el trabajo realizado por las distintas Secciones Organizativas. El Coordinador de actividades se considerará, en la Junta Directiva, representante de los intereses de las Secciones Organizativas. \item Cualquier otra estipulada por el reglamento de funcionamiento interno.

		    \end{itemize}

		    \articulo

		    El Tesorero tendrá bajo su responsabilidad el funcionamiento económico de la asociación: \begin{itemize}

		      \item Teniendo bajo su custodia todos los fondos de la asociación. \item Interviniendo con su firma todos los documentos de cobros y pagos con el conforme del Presidente. \item Llevando un libro de Estado de Cuentas con las indicaciones de ingresos, gastos, saldos y concepto, así como el inventario de los bienes materiales de la asociación si los hubiere. \item Levantando anualmente un Estado de Cuentas que se presentará al visto bueno del Presidente, además de cada vez que se requiera. \item Teniendo firma en la cuenta bancaria de la asociación. \item Llevando al corriente el pago de las cuotas de los socios, si las hubiera. \item Redactando los presupuestos, los estados de cuentas y los balances. \item Realizando cualquier otra función propia de su cargo. \item Cualquier otra estipulada por el reglamento de funcionamiento interno.

		      \end{itemize}

		      \articulo

		      El Vocal se considerará, en la Junta Directiva, representante directo de las opiniones de los asociados. Además de esta función, deberá cumplir con cualquier otra que el reglamento de funcionamiento interno le otorgue.

		      \section*{\titulo{Relaciones con otras Asociaciones}}

		      \articulo

		      La junta directiva podrá acordar: \begin{itemize}

		        \item La absorción o asociación con otras entidades de fines similares. La absorción sólo se podrá dar entre organizaciones del mismo ámbito de actuación. \item La aprobación de los acuerdos que liguen a la asociación a otras asociaciones o la anulación de dichos acuerdos.

			\end{itemize}

			\section*{\titulo{Socios}}

			\articulo

			Cualquier persona física puede adquirir la categoría de socio.

			\articulo

			La pérdida de la condición de socio, se puede producir por una o varias de las condiciones siguientes: \begin{itemize}

			  \item Por deseo expreso del asociado por medio de una carta o cualquier otro medio válidamente aceptado por la junta directiva y dirigida al Secretario o Presidente de la misma. \item Por perjudicar en sus actos gravemente los intereses o acuerdos de la asociación, a criterio de la mayoría de la junta directiva, pudiendo el socio recurrir tal decisión ante la Asamblea General en su próxima reunión. Hasta dicha Asamblea se mantendrá la condón de socio expulsado. \item Por no satisfacer la cuota de la asociación si la hubiere. \item En caso de que el reglamento de funcionamiento interno así lo recoja, ratificar periódicamente su deseo de mantenerse asociado.

			  \end{itemize}

			  \articulo

			  Constituyen derechos de todos los miembros de la asociación: \begin{itemize}

			    \item Formar parte de las Asambleas Generales con voz, y siempre que sean mayores de edad, voto. \item Poder ser elegidos miembros de la junta directiva de la asociación, siempre que sean mayores de edad. \item Disfrutar de todos los beneficios de la asociación. \item Pedir y recibir información relativa a la asociación. \item Participar en las actividades que organice la asociación. \item Recurrir ante la Asamblea General los acuerdos de la junta directiva que, en su opinión, sean contrarios a los presentes Estatutos o lesionen los intereses de la asociación

			    \end{itemize}

			    \articulo

			    Constituyen deberes de los miembros de la asociación: \begin{itemize}

			      \item Respetar los presentes Estatutos y el Reglamento de Funcionamiento Interno que determine la Asociación. \item Satisfacer en su tiempo y forma las  cuotas, si las hubiere.

			      \end{itemize}

			      \section*{\titulo{Financiación de la asociación}}

			      \articulo

			      La asociación como sujeto de derecho y para el cumplimiento de sus fines, tiene responsabilidad jurídica y titularidad de derechos y obligaciones para toda clase de actos y contratos del tráfico civil.

			      \articulo

			      La asociación podrá tener patrimonio propio y funcionará en régimen de Presupuesto Anual.

			      \articulo 

			      El ejercicio asociativo se cerrará el 1 de noviembre.

			      \articulo

			      Los medios económicos para atender sus fines serán los siguientes: \begin{itemize}

			        \item Las aportaciones voluntarias. \item Los ingresos del patrimonio que pueda poseer. \item Los donativos y/o subvenciones públicas o privadas que se le puedan conceder. \item Herencias y legados con los que pueda ser favorecida. \item Las cuotas de los socios si las hubiere. \item Otros que pudiera ingresar respetando el carácter y los fines de la asociación.

				\end{itemize}

				\articulo

				La junta directiva confeccionará todos los años un proyecto de presupuesto que presentará a la aprobación de la Asamblea General. Así mismo, presentará la liquidación de cuentas del año anterior para su aprobación.

				\articulo

				Todo lo relativo a la infraestructura de la organización y que no se considere en los presentes Estatutos, podrá ser regulado en el Reglamento de Funcionamiento Interno.

				\section*{\titulo{Disolución}}

				\articulo

				La asociación se disolverá a propuesta unánime de los miembros que pertenezcan a la Junta  Directiva y aprobación por tres cuartos de los asistentes a Asamblea General Extraordinaria al efecto.

				\articulo

				Acordada o decretada la disolución de la asociación, la Asamblea General nombrará un Comisión Liquidadora compuesta por, al menos, tres miembros elegidos por la Asamblea General Extraordinaria.

				\articulo

				La Comisión Liquidadora procederá a efectuar el inventario y valoración de los bienes de la asociación, y a su liquidación en el plazo más breve posible, en todo caso inferior a seis meses, o en el que establezca la legislación vigente para ese supuesto. Deberá dar cuenta periódicamente de su gestión, ante la junta directiva.

				\articulo

				El activo disponible, si lo hubiera, se destinará a fines benéficos o actividades culturales que estén de acuerdo con los fines de la asociación.

				\section*{\titulo{Elaboración de RFI}}

				\articulo

				La elaboración del reglamento de funcionamiento interno es competencia de la Junta Directiva en colaboración con las Secciones Organizativas o grupos correspondientes, en su totalidad o mediante un representante democráticamente escogido.

				\articulo

				Para su aprobación se requiere unanimidad favorable de la Junta Directiva.

				\articulo

				No se podrán ampliar mediante el reglamento de funcionamiento interno materias expuestas en estos Estatutos que sean competencia de la Junta Directiva.

				\subsection*{Disposición transitoria} Serán considerados socios hasta la primera renovación social todas las personas apuntadas en el libro de socios en la fecha de entrada en vigor de los presentes Estatutos.

				\subsection*{Disposición final} Los presentes estatutos entrarán en vigor en el momento de su aprobación por la Asamblea General. \end{document}

